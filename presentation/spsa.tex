% Options for packages loaded elsewhere
\PassOptionsToPackage{unicode}{hyperref}
\PassOptionsToPackage{hyphens}{url}
%
\documentclass[
  ignorenonframetext,
  aspectratio=169]{beamer}
\usepackage{pgfpages}
\setbeamertemplate{caption}[numbered]
\setbeamertemplate{caption label separator}{: }
\setbeamercolor{caption name}{fg=normal text.fg}
\beamertemplatenavigationsymbolsempty
% Prevent slide breaks in the middle of a paragraph
\widowpenalties 1 10000
\raggedbottom
\setbeamertemplate{part page}{
  \centering
  \begin{beamercolorbox}[sep=16pt,center]{part title}
    \usebeamerfont{part title}\insertpart\par
  \end{beamercolorbox}
}
\setbeamertemplate{section page}{
  \centering
  \begin{beamercolorbox}[sep=12pt,center]{part title}
    \usebeamerfont{section title}\insertsection\par
  \end{beamercolorbox}
}
\setbeamertemplate{subsection page}{
  \centering
  \begin{beamercolorbox}[sep=8pt,center]{part title}
    \usebeamerfont{subsection title}\insertsubsection\par
  \end{beamercolorbox}
}
\AtBeginPart{
  \frame{\partpage}
}
\AtBeginSection{
  \ifbibliography
  \else
    \frame{\sectionpage}
  \fi
}
\AtBeginSubsection{
  \frame{\subsectionpage}
}
\usepackage{lmodern}
\usepackage{amssymb,amsmath}
\usepackage{ifxetex,ifluatex}
\ifnum 0\ifxetex 1\fi\ifluatex 1\fi=0 % if pdftex
  \usepackage[T1]{fontenc}
  \usepackage[utf8]{inputenc}
  \usepackage{textcomp} % provide euro and other symbols
\else % if luatex or xetex
  \usepackage{unicode-math}
  \defaultfontfeatures{Scale=MatchLowercase}
  \defaultfontfeatures[\rmfamily]{Ligatures=TeX,Scale=1}
\fi
\usetheme[]{Berlin}
% Use upquote if available, for straight quotes in verbatim environments
\IfFileExists{upquote.sty}{\usepackage{upquote}}{}
\IfFileExists{microtype.sty}{% use microtype if available
  \usepackage[]{microtype}
  \UseMicrotypeSet[protrusion]{basicmath} % disable protrusion for tt fonts
}{}
\makeatletter
\@ifundefined{KOMAClassName}{% if non-KOMA class
  \IfFileExists{parskip.sty}{%
    \usepackage{parskip}
  }{% else
    \setlength{\parindent}{0pt}
    \setlength{\parskip}{6pt plus 2pt minus 1pt}}
}{% if KOMA class
  \KOMAoptions{parskip=half}}
\makeatother
\usepackage{xcolor}
\IfFileExists{xurl.sty}{\usepackage{xurl}}{} % add URL line breaks if available
\IfFileExists{bookmark.sty}{\usepackage{bookmark}}{\usepackage{hyperref}}
\hypersetup{
  pdftitle={Turnout and Amendment 4},
  pdfauthor={Kevin Morris},
  hidelinks,
  pdfcreator={LaTeX via pandoc}}
\urlstyle{same} % disable monospaced font for URLs
\newif\ifbibliography
\setlength{\emergencystretch}{3em} % prevent overfull lines
\providecommand{\tightlist}{%
  \setlength{\itemsep}{0pt}\setlength{\parskip}{0pt}}
\setcounter{secnumdepth}{-\maxdimen} % remove section numbering
\newlength{\cslhangindent}
\setlength{\cslhangindent}{1.5em}
\newenvironment{cslreferences}%
  {\setlength{\parindent}{0pt}%
  \everypar{\setlength{\hangindent}{\cslhangindent}}\ignorespaces}%
  {\par}

\title{Turnout and Amendment 4}
\subtitle{Mobilizing Eligible Voters Close to Formerly Incarcerated
Floridians}
\author{Kevin Morris}
\date{Annual Meeting of the Southern Political Science Association,
2021}
\institute{Brennan Center for Justice}

\begin{document}
\frame{\titlepage}
\begin{abstract}
\href{mailto:kevin.morris@nyu.edu}{\nolinkurl{kevin.morris@nyu.edu}}
\end{abstract}

\begin{frame}{Introduction}
\protect\hypertarget{introduction}{}
\begin{itemize}[<+->]
\tightlist
\item
  The carceral state as a site of political socialization.
\end{itemize}

\begin{itemize}[<+->]
\tightlist
\item
  Overview of the Second Chances Florida campaign in support of
  Amendment 4.
\end{itemize}

\begin{itemize}[<+->]
\tightlist
\item
  Electoral behavior of neighborhoods home to formerly incarcerated
  individuals.
\end{itemize}

\begin{itemize}[<+->]
\tightlist
\item
  Turnout among household members of the formerly incarcerated.
\end{itemize}

\begin{itemize}[<+->]
\tightlist
\item
  Post-passage legislation and litigation.
\end{itemize}
\end{frame}

\begin{frame}{Literature and Theory}
\protect\hypertarget{literature-and-theory}{}
\begin{itemize}[<+->]
\tightlist
\item
  Work from Amy Lerman and Wesla Weaver (2010; 2014) explores the
  effects of the carceral state, arguing that criminal justice contact
  teaches individuals that government is something that is ``done to''
  them, not something they are invited to participate in.
\end{itemize}

\begin{itemize}[<+->]
\tightlist
\item
  Importantly, this socialization extends also to non-convicted family
  members (Comfort 2008; Lee, Porter, and Comfort 2014; Kirk 2016).
\end{itemize}
\end{frame}

\begin{frame}{Literature and Theory}
\protect\hypertarget{literature-and-theory-1}{}
\begin{itemize}[<+->]
\tightlist
\item
  Proximal contact with the criminal justice system probably reduces
  turnout even for eligible voters (e.g.~Burch 2013; Morris 2020; but
  see White 2019).
\end{itemize}

\begin{itemize}[<+->]
\tightlist
\item
  \textbf{Research Question:} Can a contest of particular salience to
  these eligible voters --- such as Amendment 4 --- recoup their
  turnout?
\end{itemize}
\end{frame}

\begin{frame}{Key Elements of Amendment 4 Campaign}
\protect\hypertarget{key-elements-of-amendment-4-campaign}{}
\begin{itemize}[<+->]
\tightlist
\item
  Used a framework of injustice, which Walker (2020) and others indicate
  can be mobilizing.
\end{itemize}

\begin{itemize}[<+->]
\tightlist
\item
  Gillum spoke openly about the disenfranchisement of his brother; this
  potential descriptive representation could also have been mobilizing
  (e.g.~Merolla, Sellers, and Fowler 2013).
\end{itemize}
\end{frame}

\begin{frame}{Data and Methods}
\protect\hypertarget{data-and-methods}{}
\begin{itemize}[<+->]
\tightlist
\item
  Primarily using the Florida registered voter file and the Offender
  Based Information System (OBIS)
\end{itemize}

\begin{itemize}[<+->]
\tightlist
\item
  Geocoded OBIS records show the neighborhoods home to formerly
  incarcerated individuals.
\end{itemize}

\begin{itemize}[<+->]
\tightlist
\item
  By matching (cleaned) addresses in the voter file and OBIS I identify
  registered voters who live with formerly incarcerated individuals.
  These individuals are genetically matched to untreated voters, and I
  run a difference-in-differences model.
\end{itemize}
\end{frame}

\begin{frame}{Neighborhood Turnout}
\protect\hypertarget{neighborhood-turnout}{}
\begin{center}\includegraphics[width=0.9\linewidth,height=0.85\textheight]{../temp/p2} \end{center}
\end{frame}

\begin{frame}{Support for Amendment 4}
\protect\hypertarget{support-for-amendment-4}{}
\begin{center}\includegraphics[width=0.9\linewidth,height=0.85\textheight]{../temp/p1} \end{center}
\end{frame}

\begin{frame}{Roll-Off}
\protect\hypertarget{roll-off}{}
\begin{center}\includegraphics[width=0.9\linewidth,height=0.85\textheight]{../temp/p3} \end{center}
\end{frame}

\begin{frame}{Individual Level Results}
\protect\hypertarget{individual-level-results}{}
\begin{center}\includegraphics[width=0.85\linewidth,height=0.85\textheight]{../temp/comp_bars} \end{center}
\end{frame}

\begin{frame}{Post-Match Difference-in-Differences}
\protect\hypertarget{post-match-difference-in-differences}{}
\begin{center}\includegraphics[width=0.9\linewidth,height=0.9\textheight]{../temp/p} \end{center}
\end{frame}

\begin{frame}{Regression Results}
\protect\hypertarget{regression-results}{}
\begin{center}\includegraphics[width=0.85\linewidth,height=0.85\textheight]{../temp/coef1} \end{center}
\end{frame}

\begin{frame}{Regression Results}
\protect\hypertarget{regression-results-1}{}
\begin{center}\includegraphics[width=0.85\linewidth,height=0.85\textheight]{../temp/coef2} \end{center}
\end{frame}

\begin{frame}{Regression Results}
\protect\hypertarget{regression-results-2}{}
\begin{center}\includegraphics[width=0.85\linewidth,height=0.85\textheight]{../temp/coef3} \end{center}
\end{frame}

\begin{frame}{SB 7066 and Litigation}
\protect\hypertarget{sb-7066-and-litigation}{}
\begin{itemize}[<+->]
\tightlist
\item
  In the months following the passage of Amendment 4, the Florida
  legislature re-defined what it means to ``complete one's sentence.''
\end{itemize}

\begin{itemize}[<+->]
\tightlist
\item
  The state is incapable of knowing who has paid off their LFOs --- and,
  therefore, is eligible to vote.
\end{itemize}

\begin{itemize}[<+->]
\tightlist
\item
  It remains to be seen how restrictive legislation will shape these
  individuals' participation, but it is likely to re-inforce the
  negative political socialization associated with the carceral state.
\end{itemize}
\end{frame}

\begin{frame}{Thanks!}
\protect\hypertarget{thanks}{}
\href{mailto:kevin.morris@nyu.edu}{\nolinkurl{kevin.morris@nyu.edu}}
\end{frame}

\begin{frame}[allowframebreaks,allowframebreaks]{References}
\protect\hypertarget{references}{}
\hypertarget{refs}{}
\begin{cslreferences}
\leavevmode\hypertarget{ref-Burch2013}{}%
Burch, Traci R. 2013. ``Effects of Imprisonment and Community
Supervision on Neighborhood Political Participation in North Carolina:''
\emph{The ANNALS of the American Academy of Political and Social
Science}, November. \url{https://doi.org/10.1177/0002716213503093}.

\leavevmode\hypertarget{ref-Comfort2008}{}%
Comfort, Megan. 2008. \emph{Doing Time Together: Love and Family in the
Shadow of the Prison}. Chicago: University of Chicago Press.

\leavevmode\hypertarget{ref-Kirk2016}{}%
Kirk, David S. 2016. ``Prisoner Reentry and the Reproduction of Legal
Cynicism.'' \emph{Social Problems} 63 (2): 222--43.
\url{https://doi.org/10.1093/socpro/spw003}.

\leavevmode\hypertarget{ref-Lee2014}{}%
Lee, Hedwig, Lauren C. Porter, and Megan Comfort. 2014. ``Consequences
of Family Member Incarceration: Impacts on Civic Participation and
Perceptions of the Legitimacy and Fairness of Government.'' \emph{The
ANNALS of the American Academy of Political and Social Science} 651 (1):
44--73. \url{https://doi.org/10.1177/0002716213502920}.

\leavevmode\hypertarget{ref-Lerman2014}{}%
Lerman, Amy E., and Vesla M. Weaver. 2014. \emph{Arresting Citizenship:
The Democratic Consequences of American Crime Control}. Chicago Studies
in American Politics. Chicago ; London: The University of Chicago Press.

\leavevmode\hypertarget{ref-Merolla2013}{}%
Merolla, Jennifer L., Abbylin H. Sellers, and Derek J. Fowler. 2013.
``Descriptive Representation, Political Efficacy, and African Americans
in the 2008 Presidential Election.'' \emph{Political Psychology} 34 (6):
863--75. \url{https://doi.org/10.1111/j.1467-9221.2012.00934.x}.

\leavevmode\hypertarget{ref-Morris2020}{}%
Morris, Kevin. 2020. ``Neighborhoods and Felony Disenfranchisement: The
Case of New York City.'' \emph{Urban Affairs Review}, May,
1078087420921522. \url{https://doi.org/10.1177/1078087420921522}.

\leavevmode\hypertarget{ref-Walker2020}{}%
Walker, Hannah L. 2020. ``Targeted: The Mobilizing Effect of Perceptions
of Unfair Policing Practices.'' \emph{The Journal of Politics} 82 (1):
119--34. \url{https://doi.org/10.1086/705684}.

\leavevmode\hypertarget{ref-Weaver2010}{}%
Weaver, Vesla M., and Amy E. Lerman. 2010. ``Political Consequences of
the Carceral State.'' \emph{American Political Science Review} 104 (4):
817--33. \url{https://doi.org/10.1017/S0003055410000456}.

\leavevmode\hypertarget{ref-White2019a}{}%
White, Ariel. 2019. ``Family Matters? Voting Behavior in Households with
Criminal Justice Contact.'' \emph{American Political Science Review} 113
(2): 607--13. \url{https://doi.org/10.1017/S0003055418000862}.
\end{cslreferences}
\end{frame}

\end{document}
